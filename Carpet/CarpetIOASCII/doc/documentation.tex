\documentclass{article}

% Use the Cactus ThornGuide style file
% (Automatically used from Cactus distribution, if you have a 
%  thorn without the Cactus Flesh download this from the Cactus
%  homepage at www.cactuscode.org)
\usepackage{../../../../doc/ThornGuide/cactus}

\begin{document}

% The author of the documentation
\author{Erik Schnetter \textless schnetter@aei.mpg.de\textgreater\\
        Thomas Radke \textless tradke@aei.mpg.de\textgreater}

% The title of the document (not necessarily the name of the Thorn)
\title{CarpetIOASCII}

% the date your document was last changed, if your document is in CVS, 
% please use:
\date{November 20, 2005}

\maketitle

% Do not delete next line
% START CACTUS THORNGUIDE

\newcommand{\ThisThorn}{{\it CarpetIOASCII}}

%%%%%%%%%%%%%%%%%%%%%%%%%%%%%%%%%%%%%%%%%%%%%%%%%%%%%%%%%%%%%%%%%%%%%%%%%%%%
\begin{abstract}
  This thorn reproduces thorn IOASCII from arrangement CactusBase but
  is specifically for the driver thorn Carpet.
\end{abstract}


%%%%%%%%%%%%%%%%%%%%%%%%%%%%%%%%%%%%%%%%%%%%%%%%%%%%%%%%%%%%%%%%%%%%%%%%%%%%
\section{Introduction}

This thorn provides ASCII output of data in 1, 2 or 3 dimensions. It
reproduces most of the functionality of thorn IOASCII from the
standard CactusBase arrangement. Where possible the names of
parameters and their use is identical. For most purposes it should be
sufficient to take a parameter file written for the standard IOASCII
and just change the active thorn.

However, this thorn outputs considerably more information than the
standard IOASCII thorn. Information about, e.g., the refinement level
and the index position of the output are also given. All the output
can be visualized using gnuplot.

The \ThisThorn\ I/O methods can output any type of CCTK grid variables
(grid scalars, grid functions, and grid arrays of arbitrary dimension);
data is written into separate files named {\tt "<varname>.asc"}.


%%%%%%%%%%%%%%%%%%%%%%%%%%%%%%%%%%%%%%%%%%%%%%%%%%%%%%%%%%%%%%%%%%%%%%%%%%%%
\section{\ThisThorn\ Parameters}

The most important parameters to control the \ThisThorn\ I/O methods are listed
below. Many parameters come as a set of four individuals, one for each
I/O method, controlling 0, 1, 2, or 3-dimensional ASCII output respectively.

\begin{itemize}
  \item {\tt IOASCII::out[0-3]D\_every} (steerable)\\
        How often to do periodic \ThisThorn\ output. If this parameter is set
        in the parameter file, it will override the setting of the shared
        {\tt IO::out\_every} parameter. The output frequency can also be set
        for individual variables using the {\tt out\_every} option in an
        option string appended to the {\tt IOASCII::out[0-3]D\_vars} parameter.
  
  \item {\tt IOASCII::out[0-3]D\_dt} (steerable)\\
        output in intervals of that much coordinate time (overwrites
        {\tt IO::out\_dt})
  
  \item {\tt IOASCII::out[0-3]D\_criterion} (steerable)\\
        criterion to select output intervals (overwrites
        {\tt IO::out\_criterion})
  
  \item {\tt IOASCII::out[0-3]D\_vars} (steerable)\\ 
        The list of variables to output using the \ThisThorn\ I/O method.
        The variables must be given by their fully qualified variable or group
        name. The special keyword {\it all} requests \ThisThorn\ output for
        all variables. Multiple names must be separated by whitespaces.

        Each group/variable name can have an option string attached in which you
        can specify a different output frequency for that individual variable,
        or a set of individual refinement levels to be output, eg.
\begin{verbatim}
  IOASCII::out1D_vars = "wavetoy::phi{ out_every = 4 refinement_levels = { 1 2 } }"
\end{verbatim}
  
  \item {\tt IOASCII::out[0-3]D\_dir}\\
        The directory in which to place the \ThisThorn\ output files.
        If the directory doesn't exist at startup it will be created.\\
        If this parameter is set to an empty string \ThisThorn\ output will go
        to the standard output directory as specified in {\tt IO::out\_dir}.
  
  \item {\tt IOASCII::out\_precision} (steerable)\\
        How many digits to output floating-point numbers with (overwrites
        {\tt IO::out\_precision}).

  \item {\tt IOASCII::one\_file\_per\_group}\\
        Write one output file per group instead of per variable.

\end{itemize}


%%%%%%%%%%%%%%%%%%%%%%%%%%%%%%%%%%%%%%%%%%%%%%%%%%%%%%%%%%%%%%%%%%%%%%%%%%%%
\section{Using the flesh I/O API to produce ASCII output}

Periodic output of grid variables is usually specified via I/O parameters
in the parameter file and then automatically triggered by the flesh scheduler
at each iteration step after analysis. If output should also be triggered
at a different time, one can do that from within an application thorn by
invoking one of the {\tt CCTK\_OutputVar*()} I/O routines provided
by the flesh I/O API (see chapter B8.2 ``IO'' in the Cactus Users Guide).
In this case, the application thorn routine which calls {\tt CCTK\_OutputVar*()}
must be scheduled in level mode.

If output for a variable is required also for intermediate timesteps
this can be achieved by calling {\tt CCTK\_OutputVarAs*()} with a different
{\tt alias} name; output for the same variable is then written into
different ASCII files based on the {\tt alias} argument.


%%%%%%%%%%%%%%%%%%%%%%%%%%%%%%%%%%%%%%%%%%%%%%%%%%%%%%%%%%%%%%%%%%%%%%%%%%%%
\section{Utilities}
\label{sec:utils}

For those that prefer other visualization packages than gnuplot there
are some utilities in the src/util directory. These are
\begin{itemize}
\item {\bf carpet2sdf} A program to convert to a format suitable for
  the {\it ser} program written by M. Choptuik,
\item {\bf carpet2xgraph} A program to convert to a format suitable
  for the {\it xgraph} or {\it ygraph} packages of P. Walker and D.
  Pollney,
\item {\bf Carpet2ygraph.pl} A perl script to convert to a format
  suitable for the {\it xgraph} or {\it ygraph} packages of P. Walker
  and D. Pollney.
\item {\bf Carpet2ygraph.pl} An improved version of the previous script.
\item {\bf mergeCarpetIOASCII.pl} A perl script to remove duplicate datasets from file(s).
\end{itemize}

The first two, written by Scott Hawley, are C codes that require the
Makefile (building using the -utils flag from the main Cactus
directory currently does not work). Each output one refinement level,
either to standard output or to a file.

The third script writes all refinement levels from a given file in a
given direction to a number of different files given a prefix
filename, where the number in the output filename is given by the
refinement level. Usage: {\tt Carpet2ygraph.pl direction Inputfile Outputfile},
where direction is 0, 1, 2, for x,y and z.

The fourth script is an improved version of the previous one. It reads a 1D .asc file and produces a
single file containing all the data from all levels, writing points that are in more levels only
once. The output filename is the original filename with the suffix .asc changed into .xg. The
direction (x,y or z) of the input file is automatically detected from the filename. It recognizes
and converts also scalar data, like norms and extrema. Usage: {\tt Carpet2ygraphCat.pl Inputfile}.

The fifth script can be used to merge CarpetIOASCII output written before and after recovery. It
reads one or more files in CarpetIOASCII format and writes their contents to STDOUT, eliminating
duplicate datasets (all but the last occurance are discarded), which may be created when a run died
abruptly without checkpointing at the last output itaration and is recovered from a previous
checkpoint.  
Usage: {\tt mergeCarpetIOASCII.pl list of files}. \\
Example 1: {\tt mergeCarpetIOASCII.pl alp.x.asc > alp.x.asc.merged}.\\ 
Example 2: {\tt mergeCarpetIOASCII.pl alp.x.asc.firstCHKPT alp.x.asc.secondCHKPT > alp.x.asc.merged}.\\
This script was written by Thomas Radke.


%%%%%%%%%%%%%%%%%%%%%%%%%%%%%%%%%%%%%%%%%%%%%%%%%%%%%%%%%%%%%%%%%%%%%%%%%%%%
%\begin{thebibliography}{9}
%
%\end{thebibliography}

% Do not delete next line
% END CACTUS THORNGUIDE

\end{document}
