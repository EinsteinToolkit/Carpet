% *======================================================================*
%  Cactus Thorn template for ThornGuide documentation
%  Author: Ian Kelley
%  Date: Sun Jun 02, 2002
%  $Header: /home/eschnett/C/carpet/Carpet/Carpet/CarpetIOASCII/doc/documentation.tex,v 1.1 2002/10/16 12:36:05 schnetter Exp $                                                             
%
%  Thorn documentation in the latex file doc/documentation.tex 
%  will be included in ThornGuides built with the Cactus make system.
%  The scripts employed by the make system automatically include 
%  pages about variables, parameters and scheduling parsed from the 
%  relevent thorn CCL files.
%  
%  This template contains guidelines which help to assure that your     
%  documentation will be correctly added to ThornGuides. More 
%  information is available in the Cactus UsersGuide.
%                                                    
%  Guidelines:
%   - Do not change anything before the line
%       % STARTx CACTUS THORNGUIDE",
%     except for filling in the title, author, date etc. fields.
%        - Each of these fields should only be on ONE line.
%        - Author names should be sparated with a \\ or a comma
%   - You can define your own macros are OK, but they must appear after
%     the STARTx CACTUS THORNGUIDE line, and do not redefine standard 
%     latex commands.
%   - To avoid name clashes with other thorns, 'labels', 'citations', 
%     'references', and 'image' names should conform to the following 
%     convention:          
%       ARRANGEMENT_THORN_LABEL
%     For example, an image wave.eps in the arrangement CactusWave and 
%     thorn WaveToyC should be renamed to CactusWave_WaveToyC_wave.eps
%   - Graphics should only be included using the graphix package. 
%     More specifically, with the "includegraphics" command. Do
%     not specify any graphic file extensions in your .tex file. This 
%     will allow us (later) to create a PDF version of the ThornGuide
%     via pdflatex. |
%   - References should be included with the latex "bibitem" command. 
%   - use \begin{abstract}...\end{abstract} instead of \abstract{...}
%   - For the benefit of our Perl scripts, and for future extensions, 
%     please use simple latex.     
%
% *======================================================================* 
% 
% Example of including a graphic image:
%    \begin{figure}[ht]
%       \begin{center}
%          \includegraphics[width=6cm]{MyArrangement_MyThorn_MyFigure}
%       \end{center}
%       \caption{Illustration of this and that}
%       \label{MyArrangement_MyThorn_MyLabel}
%    \end{figure}
%
% Example of using a label:
%   \label{MyArrangement_MyThorn_MyLabel}
%
% Example of a citation:
%    \cite{MyArrangement_MyThorn_Author99}
%
% Example of including a reference
%   \bibitem{MyArrangement_MyThorn_Author99}
%   {J. Author, {\em The Title of the Book, Journal, or periodical}, 1 (1999), 
%   1--16. {\tt http://www.nowhere.com/}}
%
% *======================================================================* 

% If you are using CVS use this line to give version information
% $Header: /home/eschnett/C/carpet/Carpet/Carpet/CarpetIOASCII/doc/documentation.tex,v 1.1 2002/10/16 12:36:05 schnetter Exp $

\documentclass{article}

% Use the Cactus ThornGuide style file
% (Automatically used from Cactus distribution, if you have a 
%  thorn without the Cactus Flesh download this from the Cactus
%  homepage at www.cactuscode.org)
\usepackage{../../../../doc/ThornGuide/cactus}

\begin{document}

% The author of the documentation
\author{Erik Schnetter} 

% The title of the document (not necessarily the name of the Thorn)
\title{CarpetIOASCII}

% the date your document was last changed, if your document is in CVS, 
% please use:
\date{$ $Date: 2002/10/16 12:36:05 $ $}
%\date{}

\maketitle

% Do not delete next line
% START CACTUS THORNGUIDE

% Add all definitions used in this documentation here 
%   \def\mydef etc

% Add an abstract for this thorn's documentation
\begin{abstract}
  This thorn reproduces thorn IOASCII from arrangement CactusBase but
  is specifically for the driver thorn Carpet.
\end{abstract}

% The following sections are suggestive only.
% Remove them or add your own.

\section{Introduction}

This thorn provides ASCII output of data in 1, 2 or 3 dimensions. It
reproduces most of the functionality of thorn IOASCII from the
standard CactusBase arrangement. Where possible the names of
parameters and their use is identical. For most purposes it should be
sufficient to take a parameter file written for the standard IOASCII
and just change the active thorn.

However, this thorn outputs considerably more information than the
standard IOASCII thorn. Information about, e.g., the refinement level
and the index position of the output are also given. All the output
can be visualized using gnuplot.

\section{Utilities}
\label{sec:utils}

For those that prefer other visualization packages than gnuplot there
are some utilities in the src/util directory. These are
\begin{itemize}
\item {\bf carpet2sdf} A program to convert to a format suitable for
  the {\it ser} program written by M. Choptuik,
\item {\bf carpet2xgraph} A program to convert to a format suitable
  for the {\it xgraph} or {\it ygraph} packages of P. Walker and D.
  Pollney,
\item {\bf Carpet2ygraph.pl} A perl script to convert to a format
  suitable for the {\it xgraph} or {\it ygraph} packages of P. Walker
  and D. Pollney.
\end{itemize}
The first two, written by Scott Hawley, are C codes that require the
Makefile (building using the -utils flag from the main Cactus
directory currently does not work). Each output one refinement level,
either to standard output or to a file.

The last script writes all refinement levels from a given file in a
given direction to a number of different files given a prefix
filename, where the number in the output filename is given by the
refinement level.


\begin{thebibliography}{9}

\end{thebibliography}

% Do not delete next line
% END CACTUS THORNGUIDE

\end{document}
