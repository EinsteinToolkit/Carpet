% *======================================================================*
%  Cactus Thorn template for ThornGuide documentation
%  Author: Ian Kelley
%  Date: Sun Jun 02, 2002
%  $Header: /home/eschnett/C/carpet/Carpet/Carpet/CarpetIOHDF5/doc/documentation.tex,v 1.3 2004/03/18 19:52:17 cott Exp $                                                             
%
%  Thorn documentation in the latex file doc/documentation.tex 
%  will be included in ThornGuides built with the Cactus make system.
%  The scripts employed by the make system automatically include 
%  pages about variables, parameters and scheduling parsed from the 
%  relevent thorn CCL files.
%  
%  This template contains guidelines which help to assure that your     
%  documentation will be correctly added to ThornGuides. More 
%  information is available in the Cactus UsersGuide.
%                                                    
%  Guidelines:
%   - Do not change anything before the line
%       % START CACTUS THORNGUIDE",
%     except for filling in the title, author, date etc. fields.
%        - Each of these fields should only be on ONE line.
%        - Author names should be sparated with a \\ or a comma
%   - You can define your own macros are OK, but they must appear after
%     the START CACTUS THORNGUIDE line, and do not redefine standard 
%     latex commands.
%   - To avoid name clashes with other thorns, 'labels', 'citations', 
%     'references', and 'image' names should conform to the following 
%     convention:          
%       ARRANGEMENT_THORN_LABEL
%     For example, an image wave.eps in the arrangement CactusWave and 
%     thorn WaveToyC should be renamed to CactusWave_WaveToyC_wave.eps
%   - Graphics should only be included using the graphix package. 
%     More specifically, with the "includegraphics" command. Do
%     not specify any graphic file extensions in your .tex file. This 
%     will allow us (later) to create a PDF version of the ThornGuide
%     via pdflatex. |
%   - References should be included with the latex "bibitem" command. 
%   - use \begin{abstract}...\end{abstract} instead of \abstract{...}
%   - For the benefit of our Perl scripts, and for future extensions, 
%     please use simple latex.     
%
% *======================================================================* 
% 
% Example of including a graphic image:
%    \begin{figure}[ht]
% 	\begin{center}
%    	   \includegraphics[width=6cm]{MyArrangement_MyThorn_MyFigure}
% 	\end{center}
% 	\caption{Illustration of this and that}
% 	\label{MyArrangement_MyThorn_MyLabel}
%    \end{figure}
%
% Example of using a label:
%   \label{MyArrangement_MyThorn_MyLabel}
%
% Example of a citation:
%    \cite{MyArrangement_MyThorn_Author99}
%
% Example of including a reference
%   \bibitem{MyArrangement_MyThorn_Author99}
%   {J. Author, {\em The Title of the Book, Journal, or periodical}, 1 (1999), 
%   1--16. {\tt http://www.nowhere.com/}}
%
% *======================================================================* 

% If you are using CVS use this line to give version information
% $Header: /home/eschnett/C/carpet/Carpet/Carpet/CarpetIOHDF5/doc/documentation.tex,v 1.3 2004/03/18 19:52:17 cott Exp $

\documentclass{article}

% Use the Cactus ThornGuide style file
% (Automatically used from Cactus distribution, if you have a 
%  thorn without the Cactus Flesh download this from the Cactus
%  homepage at www.cactuscode.org)
\usepackage{../../../../doc/ThornGuide/cactus}

\begin{document}

% The author of the documentation
\author{Erik Schnetter $<$schnetter@uni-tuebingen.de$>$, Christian D. Ott $<$cott@aei.mpg.de$>$}

% The title of the document (not necessarily the name of the Thorn)
\title{CarpetIOHDF5}

% the date your document was last changed, if your document is in CVS, 
% please use:
%    \date{$ $Date: 2004/03/18 19:52:17 $ $}
\date{March 18, 2004}

\maketitle

% Do not delete next line
% START CACTUS THORNGUIDE

% Add all definitions used in this documentation here 
%   \def\mydef etc

% Add an abstract for this thorn's documentation
\begin{abstract}
CarpetIOHDF5 provides HDF5 based 3-D output to the Cactus mesh refinement driver Carpet.
This document explains CarpetIOHDF5's usage and contains a specification of the
CarpetIOHDF5 file format that was adapted from John Shalf's FlexIO library.
\end{abstract}

% The following sections are suggestive only.
% Remove them or add your own.

\section{Introduction}

Having encountered various problems with CarpetIOFlexIO and the underlying FlexIO library,
Erik Schnetter decided to create CarpetIOHDF5. CarpetIOHDF5 provides 3-D output for the
Carpet Mesh Refinement driver within the Cactus Code. Christian D. Ott added  
file reader (analogous to Erik Schnetter's implementation present in CarpetIOFlexIO) 
and checkpoint/recovery features to CarpetIOHDF5.

Right now, CarpetIOHDF5 only uses serial I/O - all data are copied to processor 0 for I/O.

This document aims at giving the user a first handle on how to use
CarpetIOHDF5. It also documents the HDF5 file layout used.


\section{Using This Thorn}


\subsection{Obtaining This Thorn}

You can get a checkout from 

{\tt cvs.carpetcode.org:/home/cvs Carpet/CarpetIOHDF5}

\subsection{Basic Usage}

First, you have to activate the thorn in your Cactus parameter file:

{\tt ActiveThorns = "CarpetIOHDF5"}

\subsubsection{3-D Output}

\begin{itemize}
  \item {\tt iohdf5::out3D\_vars = "your list of Cactus grid functions to output"}
  \item {\tt iohdf5::out3D\_every = n} : Output every {\tt n} time steps
  \item {\tt iohdf5::out3D\_dir = "your preferred 3-D output directory"}
\end{itemize}

\subsubsection{3-D Input}

There are two ways to use the 3-D input capabilities:

\begin{enumerate}
  \item For evolutions using ADMBase, one may use the thorn IDFileADM and the following parameter settings:
    \begin{itemize}
      \item {\tt ADMBase::initial\_data  = "read from file"}
      \item {\tt IO::filereader\_ID\_files = "space separated list of files containing the ADM variables"}
      \item {\tt IO::filereader\_ID\_vars = "space separated list of variables to be read in"}
    \end{itemize}
  \item For evolutions not using ADMBase one may try to read in data by setting
    \begin{itemize}
      \item {\tt iohdf5::in3D\_dir = "directory from where to read data"}
      \item {\tt iohdf5::in3D\_vars = "space separated list of variables to be read in"}
    \end{itemize}
\end{enumerate}


\subsubsection{Checkpointing}

CarpetIOHDF5 uses the Cactus checkpoint/recovery infrastructure provided by
CactusBase/IOUtil.

\begin{itemize}
  \item {\tt iohdf5::checkpoint = "yes"} : Turns on checkpointing
  \item {\tt iohdf5::checkpoint\_every = n} : Checkpointing every {\tt n} time steps
  \item {\tt iohdf5::checkpoint\_ID = "yes"} : Turns on checkpointing after initial data
  \item {\tt io::checkpoint\_dir = "your preferred checkpoint directory"} 
  \item  {\tt iohdf5::checkpoint\_keep = n} : Keep {\tt n} checkpoint files around
\end{itemize}


\subsubsection{Recover}

CarpetIOHDF5 uses the Cactus checkpoint/recovery infrastructure provided by
CactusBase/IOUtil. Currently all the checkpoint information is copied onto processor 0 and
 written into a single file whose name is invented by IOUtil. Unfortunately, single cpu
checkpoint files have a different name (a missing \_file\_0 tag) than checkpoint
files from multi-cpu runs. Somebody should tweek IOBase... 

In principle, CarpetIOHDF5 is able to restart on any number
of cpus  from a checkpoint file of a run using any (other or same) number of cpus.

\begin{itemize}
  \item {\tt iohdf5::recover = "auto"} : Recover from the most recent Checkpoint file. This bombs,
    if no checkpoint file is found.
  \item {\tt iohdf5::recover = "autoprobe"} : Recover from the most recent Checkpoint file. This continues
    without recovering if no checkpoint file is found.
  \item {\tt iohdf5::recover\_dir = "directory containing the checkpoint file"} 
  \item {\tt iohdf5::recover = "manual"} : Recover from a file specified by {\tt iohdf5::recover\_file}. This
     bombs if the file is not found.
  \item {\tt iohdf5::recover\_file = "file you want to recover from"} : Only needs to be set if
    {\tt iohdf5::recover = "manual"}.

\end{itemize}



\subsection{Special Behaviour}

\begin{itemize}
  \item {\tt iohdf5::h5verbose = "yes"} : Makes CarpetIOHDF5 very talkative.
\end{itemize}


\section{CarpetIOHDF5's HDF5 file layout}

The HDF5 file layout of CarpetIOHDF5 is quite simple. There are no groups besides the 
standard HDF5 root data object group:

\begin{itemize}
  \item Each dataset is named according to this template: \\
    {\tt \small [Full Variable Name] it=[Iteration] tl=[Timelevel] ml=[mglevel] rl=[reflevel] m=[map] c=[component]}
    
  \item Each dataset has the following (largely redundant!) attributes associated with it:

    \begin{itemize}
    \item {\tt level} : Carpet::reflevel
    \item {\tt origin} : 1-D array of length Carpet::dim. \\ origin[d]=CCTK\_ORIGIN\_SPACE(d) + 
      cctk\_lbnd[d] * delta[d]
    \item {\tt delta} : 1-D array of length Carpet::dim. \\ delta[d] = CCTK\_DELTA\_SPACE(d)
    \item {\tt min\_ext} : 1-D array of length Carpet::dim. \\ min\_ext[d] = delta[d] 
    \item {\tt max\_ext} : 1-D array of length Carpet::dim. \\ origin[d] + cctk\_lsh[d] * delta[d]
    \item {\tt time} : cctk\_time
    \item {\tt timestep} : cctk\_iteration
    \item {\tt level\_timestep} : cctk\_iteration / Carpet::reflevelfact
    \item {\tt persistence} : cctk\_iteration / Carpet::reflevelfact
    \item {\tt time\_refinement} : Carpet::time\_refinement
    \item {\tt spatial\_refinement} : 1-D array of length Carpet::dim. \\ spatial\_refinement[d] = Carpet::reflevelfact
    \item {\tt grid\_placement\_refinement} : 1-D array of length Carpet::dim. \\ grid\_placement\_refinement[d] = Carpet::reflevelfact
    \item {\tt iorigin} : 1-D array of length Carpet::dim. \\ iorigin[d] = (Carpet::ext.lower() / Carpet::ext.stride())[d]
    \item {\tt name} : CCTK\_FullName(variable index)
    \item {\tt group\_version} : 1
    \item {\tt group\_fullname} : CCTK\_FullName(variable index)
    \item {\tt group\_varname} : CCTK\_VarName(variable index)
    \item {\tt group\_groupname} : CCTK\_GroupName(group index)
    \item {\tt group\_grouptype} : CCTK\_GF, CCTK\_ARRAY or CCTK\_SCALAR
    \item {\tt group\_dim} : CCTK\_GroupDimI(group index)
    \item {\tt group\_timelevel} : tl (current timelevel)
    \item {\tt group\_numtimelevels} : CCTK\_NumTimeLevelsI(group index)
    \item {\tt cctk\_version} : 1
    \item {\tt cctk\_dim} : cctk\_dim
    \item {\tt cctk\_iteration} : cctk\_iteration
    \item {\tt cctk\_gsh} : 1-D array of length Carpet::dim. cctk\_gsh
    \item {\tt cctk\_lsh} : 1-D array of length Carpet::dim. cctk\_lsh
    \item {\tt cctk\_lbnd} : 1-D array of length Carpet::dim. cctk\_lbnd
    \item {\tt cctk\_delta\_time} : 1-D array of length Carpet::dim. cctk\_delta\_time
    \item {\tt cctk\_delta\_space} : 1-D array of length Carpet::dim. cctk\_delta\_space
    \item {\tt cctk\_origin\_space} : 1-D array of length Carpet::dim. cctk\_origin\_space
    \item {\tt cctk\_bbox} : 1-D array of length 2*Carpet::dim. cctk\_box
    \item {\tt cctk\_levfac} : 1-D array of length Carpet::dim. cctk\_levfac
    \item {\tt cctk\_levoff} : 1-D array of length Carpet::dim. cctk\_levoff
    \item {\tt cctk\_levoffdenom} : 1-D array of length Carpet::dim. cctk\_levoffdenom
    \item {\tt cctk\_timefac} : cctk\_timefac
    \item {\tt cctk\_convlevel} : cctk\_convlevel
    \item {\tt cctk\_convfac} : cctk\_convfac
    \item {\tt cctk\_nghostzones} : 1-D array of length Carpet::dim. cctk\_nghostzones
    \item {\tt cctk\_time} : cctk\_time
    \item {\tt carpet\_version} : 1
    \item {\tt carpet\_dim} : Carpet::dim
    \item {\tt carpet\_basemglevel} : Carpet::basemglevel
    \item {\tt carpet\_mglevel} : Carpet::mglevel
    \item {\tt carpet\_mglevels} : Carpet::mglevels
    \item {\tt carpet\_mgface} : Carpet::mgfact
    \item {\tt carpet\_reflevel} : Carpet::reflevel
    \item {\tt carpet\_reflevels} : Carpet::reflevels
    \item {\tt carpet\_reffact} : Carpet::reffact
    \item {\tt carpet\_map} : Carpet::map
    \item {\tt carpet\_maps} : Carpet::maps
    \item {\tt carpet\_component} : Carpet::component
    \item {\tt carpet\_components} : Carpet::vhh.at(Carpet::map)->components(reflevel))

    \end{itemize}
\end{itemize}

\subsection{Attributes needed by the file reader}

The number of attributes needed by the CarpetIOHDF5 file reader is much smaller then the total
number of attributes attached to each dataset:

\begin{itemize}
  \item {\tt name}
  \item {\tt level}
  \item {\tt iorigin}
\end{itemize}


%\subsection{Interaction With Other Thorns}
%
%\subsection{Support and Feedback}
%
%\section{History}
%
%\subsection{Thorn Source Code}
%
%\subsection{Thorn Documentation}
%
%\subsection{Acknowledgements}
%
%
%\begin{thebibliography}{9}
%
%\end{thebibliography}
%
% Do not delete next line
% END CACTUS THORNGUIDE

\end{document}
