% $Header: /home/eschnett/C/carpet/Carpet/Carpet/doc/internals.tex,v 1.4 2003/05/03 13:29:23 schnetter Exp $

\documentclass{article}

\usepackage{amsfonts}
\usepackage{amssymb}
\usepackage[english] {babel}
\usepackage{exscale}
\usepackage[final]{graphicx}
\usepackage[backref,draft=false]{hyperref}
%\usepackage{concrete}
\usepackage{mathpple}
%\usepackage{pslatex}

\newcommand{\todo}[1]{\rule{1em}{1ex}~{\small [{#1}]}}

\sloppypar

\begin{document}

\title{Carpet under the hood}
\author{Erik Schnetter \textless schnetter@uni-tuebingen.de\textgreater}
\date{$ $Date: 2003/05/03 13:29:23 $ $}
\maketitle

\begin{abstract}
   This document describes the internal workings of the Carpet
   arrangement.  Its intended readership are people who extend Carpet,
   or who use Carpet more thant the average user.  This document is
   supposed to be read in conjuction with and guiding through the
   source code.
\end{abstract}

\tableofcontents

\section{Overview}

   The Carpet driver, which lives in the Carpet arrangement, is
   divided into several parts.  The thorn \texttt{Carpet} is the main
   driver piece; it provides all the routines and structures that
   Cactus expects from it.  The thorn \texttt{CarpetLib} is the
   workhorse that does all the bookkeeping and data shuffling.  Those
   two alone form a valid Cactus driver; the other thorns provide
   additional functionality.  The thorns \texttt{CarpetInterp},
   \texttt{CarpetReduce}, and \texttt{CarpetSlab} provide the
   corresponding interpolation, reduction, and slabbing interfaces.
   The thorns \texttt{CarpetIOASCII} and \texttt{CarpetIOFlexIO}
   provide I/O methods.  Finally, thorn \texttt{CarpetRegrid} provides
   a user interface to select where and what to refine.  (The actual
   refinement is handled in \texttt{CarpetLib}.)



\section{Terminology}

   Carpet is called ``Carpet'' because a carpet consists of many
   individual patches.

   Carpet is a mesh refinement driver.  It knows about a hierarchy of
   \emph{refinement levels}, where each level is decomposed into a set
   of cuboid \emph{grid patches}.  For historic reasons it also has a
   notion of \emph{multigrid levels}, but those are currently unused.
   They might conceivably be reactivated to form multigrid stacks to
   solve elliptic equations.  The grid patch is the smallest unit of
   grid points that Carpet deals with.  Carpet parallelises by
   assigning sets of grid patches to processors.

   A multi-patch run is a run where more than one grid patch (of the
   same refinement level) is assigned to a single processor.  This is
   a situation that can occur even without refinement.  This is also a
   situation that cannot occur with PUGH, so that most thorns cannot
   handle this situation.  In multi-patch runs one has to distinguish
   between \emph{local mode}, where one has access to a single grid
   patch, and \emph{global mode}, where one cannot access individual
   grid patches, but can instead perfom global operations such as
   synchronisation, interpolation, or reduction.  This part of Cactus
   is currently (2003-04-30) undergoing changes.

   Carpet uses vertex-centered refinement.  That is, each coarse grid
   point coincides with a fine grid point.  To \emph{regrid} means to
   select a new set of grid patches for each refinement level.  To
   \emph{recompose} the grid hierarchy means to move data around.
   Regridding is only about bookkeeping, while recomposing is about
   data munging.

   Each grid patch can be divided in up to four zones: the interior,
   the outer boundary, and the ghost zone, and the refinement
   boundary.  The interior is where the actual compuations go on.  The
   outer boundary is where the users' outer boundary condition is
   applied; from Carpet's point of view, these two are the same.  (The
   only difference is that Carpet sets \texttt{cctk\_bbox}
   correspondingly.)  The ghost zones are boundaries to other grid
   patches on the same refinement level (that might live on a
   different processor).  The refinement boundary is the boundary of
   the refined region in a level, and it is filled by prolongation
   (interpolation) from the next coarser level.  Both the ghost zones
   and the prolongation boundary are filled by \emph{synchronising}.

   Grid patches that are on the same refinement level never overlap
   except with their ghost zones.  Conversly, all ghost zones must
   overlap with a non-ghost zone of another grid patch of the same
   level.  All refinement boundaries must overlap with a grid patch on
   the next coarser level.  (This is also called \emph{proper
   nesting}.)

   Except for exceptions, Carpet numbers grid point indices and time
   levels with integers.  It counts always in terms of the finest
   grid, so that coarser grids have \emph{strides} that are powers of
   the refinement factor.  This has the advantage that different
   refinement levels can use the same global numbering scheme.

   The grid patches are described by a \emph{bounding box}
   (abbreviated bbox, see \texttt{CarpetLib/src/bbox.*}.).  This is a
   triplet of \emph{vectors} (see \texttt{CarpetLib/src/vect.*}),
   where each triplet specifies \emph{lower bound}, \emph{upper
   bound}, and \emph{stride}, much as is conventional in Fortran.
   Triplets are enclosed in round parentheses $(\cdot:\cdot:\cdot)$,
   and vectors are enclosed in square brackets $[\cdot,\cdot,\cdots]$.
   A typical grid patch might have a bounding box which is denoted by
   $([0,0,0]:[20,20,20]:[2,2,2])$.  This is to be read as
   $(\textrm{lower}:\textrm{upper}:\textrm{stride})$, meaning that the
   grid patch has one corner grid point at $[0,0,0]$, the diagonally
   opposite corner grid point at $[20,20,20]$, and the grid points are
   spaced two ``fine grid spacings'' apart.  This grid patch contains
   $11 \times 11 \times 11$ grid points.  Empty bboxes have an upper
   bound that is strictly lower than the lower bound.  The files
   \texttt{CarpetLib/src/vect.*} contains many useful routines to deal
   with short vectors, and the files \texttt{CarpetLib/src/bbox.*}
   contain routines deal with an algebra of bboxes.  The files
   \texttt{CarpetLib/src/bboxset.*} contain routines that handle sets
   of bboxes.



\section{The driver}

   The driver consists of the two thorns \texttt{Carpet} and
   \texttt{CarpetLib}.  \texttt{Carpet} is the front end to
   Cactus, while \texttt{CarpetLib} is the back end to the
   machine.  \texttt{Carpet} specifies the grid shape, decides when to
   allocate and deallocate storage, cycles through thes schedule bins,
   and passes all internal information in the \texttt{cGH} structure
   to the thorns.



\subsection{Specifying the grid extent}

   \texttt{Carpet} defines the usual parameters necessary to specify
   the extent of the grid.  Everything that has to do with coordinates
   and symmetries is handled elsewhere, and the driver does not know
   about that.

   The \texttt{global\_*} parameters specify the global extent of the
   coarsest grid.  Not all of this grid needs to be covered by grid
   patches.  It is conceivable to have an L-shaped simulation domain
   without any refinement.  This situation can be described to Carpet
   by specifying a global shape that is the convex hull of the domain,
   and then using two cuboid grid patchs to fill in the shape of
   the~L.

   The \texttt{ghost\_*} parameters specify the number of ghost zones.
   The \texttt{periodic*} parameters are unused; they are only there
   because some thorns look at these parameters.  Carpet itself does
   not supply periodic boundary conditions; they have to be handled by
   another thorn.  The size of the prolongation boundary is the same
   as the number of ghost zones.

   The parameter \texttt{max\_refinement\_levels} specifies the
   maximum number of levels that can be present in a run, including
   the base level.  This parameter, together with
   \texttt{refinement\_factor}, define the grid point numbering
   scheme, which (see above) depends on the finest possible grid.
   However, none of the finer levels will be activated automatically.
   The \texttt{multigrid\_*} parameters are unused.

   The parameter \texttt{base\_extents} specifies the shapes of the
   grid patches that are present on the coarsest grid.  This can be
   used to set up e.g.\ an L-shaped domain.  The parameter
   \texttt{base\_outerbounds} specifies which of the grid patches'
   boundaries are to be treated as outer boundaries, i.e.\ for which
   of those \texttt{cctk\_bbox} should be set to 1.

   Carpet currently ignores \texttt{enable\_all\_storage} and always
   enables all storage.  This is because it is not yet clear how
   individual grid function can be allocated and deallocated while
   still keeping enough data for the boundary prolongation.

   Checksumming and poisoning are means to find thorns that alter grid
   variables that should not be altered, or that fail to fill in grid
   variables that they should fill in.

   None of the above specifies anything about refined grids.  Refined
   grid are created and destroyed at run time, possibly guided by the
   thorn \texttt{CarpetRegrid} which provides a nice user interface.



\subsection{The timeline}

   It is \texttt{Carpet}'s task to walk through the schedule bins and
   call all user routines.  Only some fairly fundamental
   initialisation happens in the flesh before Carpet takes control.
   The overall picture of what happens when is:
\begin{enumerate}
\item
   Startup (see file \texttt{Carpet/src/CarpetStartup.cc}).  This is
   the only scheduled routine; everything else happens by overloading
   and registering.  This routine does nothing but registering and
   overloading.
\item
   SetupGH (see file \texttt{Carpet/src/SetupGH.cc}).  This routine
   does the bulk of initialising Carpet.  It sets up the internal
   structures for all grid variables.
\item
   Initialise (see file \texttt{Carpet/src/Initialise.cc}).  This
   routine walks the initialisation part of the scheduling bins.
\item
   Evolve (see file \texttt{Carpet/src/Evolve.cc}).  This routine
   walks the evolution part of the scheduling bins.  It also contains
   the main evolution loop.
\item
   Shutdown (see file \texttt{Carpet/src/Shutdown.cc}).  This routine
   walks the shutdown part of the scheduling bins.  Normally, nothing
   interesting happens here.
\end{enumerate}
   These stages are explained in the following sections.



\subsubsection{Initialisation}

   (See file \texttt{Carpet/src/Initialise.cc}.)  In this stage Carpet
   initialises the simulation.  This includes setting up the grids,
   calling routines to register symmetries and boundary conditions, as
   well as calculating the actual initial data on several refinement
   levels.  It traverses the scheduling bins in the following order:
\begin{enumerate}
\itemsep 0pt
\item
   Set \texttt{cctk\_iteration} to zero
\item
   Set \texttt{cctk\_time} to the initial time
\item
   PARAMCHECK
\item
   Loop over refinement levels, starting from coarsest:
\item \quad
   BASEGRID
\item \quad
   INITIAL
\item \quad
   POSTINITIAL
\item \quad
   POSTSTEP
\item \quad
   Regrid (possibly creating new levels)
\item
   End loop over refinement levels
\item
   Restrict from finer to coarser grids
\item
   If desired, perform Scott Hawley's initialisation scheme for three
   timelevels
\item
   Loop over refinement levels, starting from coarsest:
\item \quad
   RECOVER\_VARIABLES
\item \quad
   CPINITIAL
\item \quad
   ANALYSIS
\item \quad
   OutputGH
\item
   End loop over refinement levels
\end{enumerate}

   In the beginning, only the coarsest level exists.  The first loop
   starts by initialising this level.  At the end of this loop, more
   levels are created if desired.  This makes it possible to make this
   decision depend on an automatic refinement criterion.



\subsubsection{Evolution}

   (See file \texttt{Carpet/src/Evolve.cc}.)  In this stage Carpet
   performs the main time evolution loop.  This is further complicated
   by the fact that finer grids need to take more and smaller time
   steps than coarser grids.  In Carpet's time step counting scheme,
   which is based on the finest grid time steps, this means that the
   coarser grids are skipped in the remaining time steps.  Thus the
   elegant recursive scheme is flattened out.  The scheduling bins in
   the main time evolution loop are traversed in the following order:
\begin{enumerate}
\itemsep 0pt
\item
   Advance \texttt{cctk\_iteration}
\item
   Loop over refinement levels, starting from coarsest:
\item \quad
   If the current level needs to be treated at this iteration:
\item \quad \quad
   Calculate current \texttt{cctk\_time}
\item \quad \quad
   Cycle time levels
\item \quad \quad
   PRESTEP
\item \quad \quad
   EVOL
\item \quad \quad
   POSTSTEP
\item \quad \quad
   Regrid
\item
   End loop over refinement levels
\item
   Restrict from finer to coarser grids
\item
   Loop over refinement levels, starting from coarsest:
\item \quad
   If the current level needs to be treated at this iteration:
\item \quad \quad
   CHECKPOINT
\item \quad \quad
   ANALYSIS
\item \quad \quad
   OutputGH
\item
   End loop over refinement levels
\end{enumerate}

   The condition whether a refinement level needs to be treated at the
   current iteration is different for the two loops.  In the first
   loop, the coarse grids need to be advanced before the finer grids,
   so the condition is $iter \,\mathrm{mod}\, stride = 1$.  Here
   $iter$ is the current iteration, and $stride$ the stride of the
   current refinement level, i.e.\ the factor by which the finest grid
   is finer than the current grid.  In the second loop above, the
   coarse grids need to be treated after the finer grids, so that the
   condition reads $iter \,\mathrm{mod}\, stride = stride$.



\subsection{Calling scheduled routines}

   (See file \texttt{Carpet/src/CallFunction.cc}.)  The process by
   which the scheduling bins are traversed is different from the
   process which actually calls the routines within the scheduling
   bins.  The former has to do with mesh refinement, making sure that
   the coarse and fine grids are evolved in the right order.  The
   latter has to do with treating multiple patches, i.e.\ with local
   mode and global mode operations, as mentioned above.

   In the function \texttt{CallFunction}, all the arguments that are
   passed to the scheduled routines have to be set up.  Additionally,
   the \texttt{cGH} structure has to be filled in.  Some fields in the
   \texttt{cGH} structure are always kept up-to-date during the
   refinement level loops, such as the time step size and the grid
   spacing.  The file \texttt{Carpet/src/helper.cc} contains helper
   routines that allow easy looping over refinement levels and over
   grid patches.  (Grid patches are also called \emph{compoments} in
   Carpet.  The expression component seems to be confusing, so that I
   switched to using \emph{patch} instead.  Some source code still
   reflects the old conventsion.)

   The function \texttt{CallFunction} first distinguishes between
   global mode functions and local mode functions.
\begin{description}
\item[Global mode functions]
   are called once (on each processor).  They are passed all the
   global data, such as \texttt{cctk\_gsh} and
   \texttt{cctk\_delta\_space}, but none of the local data, such as
   \texttt{cctk\_lsh} or \texttt{cctk\_bbox}.  Grid functions are not
   accessible, and they are passed as null pointers.  However, grid
   scalars and grid arrays are accessible.  There is an untested
   gateway to directly call local mode functions from global mode
   functions.
\item[Local mode functions]
   might be called several times (on each processor), once for each
   grid patch that is assigned to this processor.  They receive the
   global data as well as data for a single grid patch.  It is illegal
   to perform global operations, such as synchronisation,
   interpolation, or reduction, in local mode.
\end{description}

   The distinction between global and local mode is only important for
   multi-patch runs.  For single-patch runs, the distinction does not
   exist.

   Multi-patch runs are only necessary when there are more grid
   patches on a refinement level than there are processors.  This is
   normally not the case for fixed mesh refinement.  Things are
   different for adaptive mesh refinement, which can create many
   refined regions.



\subsection{Grid arrays and grid scalars}

   Grid scalars are implemented as zero-dimensional grid arrays with
   \texttt{DISTRIB=CONSTANT}.

   Grid arrays are implemented as grid functions, where each grid
   array group has their own refinement hierarchy that consists of a
   single level only and is never changed at run time.  Grid arrays
   with less than 3 dimension are extended to have an extent of 1 (and
   no ghost zones) in the remaining dimensions, so that all quantities
   in Carpet have 3 dimensions\footnote{This is set by a compile-time
   constant and could be changed to allow for grid functions and
   arrays with more than 3 dimensions.}.  \texttt{DISTRIB=CONSTANT} grid arrays
   are implemented by internally enlarging the grid array in the $z$
   direction, and then distributing this array onto the processors.



\subsection{Flesh interfaces}

   The flesh has many interfaces that need to be filled in by a
   driver.  These are in particular all the routines that are
   overloaded in the SetupGH stage.  Those overloaded routines as well
   as other helper function are implemented in the following files:
\begin{description}
\itemsep 0pt
\item[\texttt{Carpet/src/Checksum.cc}]
   catching illegal changes to grid variables
\item[\texttt{Carpet/src/Comm.cc}]
   synchronisation, prolongation
\item[\texttt{Carpet/src/Cycle.cc}]
   time level handling
\item[\texttt{Carpet/src/Poison.cc}]
   catching uninitialised grid variables
\item[\texttt{Carpet/src/Restrict.cc}]
   restriction from finer to coarser grids
\item[\texttt{Carpet/src/Storage.cc}]
   enabling and disabling storage
\item[\texttt{Carpet/src/helpers.cc}]
   small low-level helper routines
\item[\texttt{Carpet/src/variables.cc}]
   the global variables that keep Carpet's current state (this is used
   instead of a GH extension --- should probably be changed some time)
\end{description}

   Most of these files are fairly self-contained, and they mostly
   marshal the actual work to \texttt{CarpetLib}.



\subsection{Interfaces to other thorns}

   Some other thorns, mostly from the Carpet arrangement, do need to
   access internal data of Carpet.  Carpet keeps its internal state in
   global variables which are declared in
   \texttt{Carpet/src/carpet\_public.hh} and defined in
   \texttt{Carpet/src/variables.cc}.  Entities that can be accessed
   from C are declared in \texttt{Carpet/src/carpet\_public.h}; some
   of these would be quite useful if they were provided by the flesh.



\subsection{Missing parts}

   \texttt{Carpet} does not handle staggered grids.  \texttt{Carpet}
   does not provide cell-centered refinement.  \texttt{Carpet} always
   enables all storage.  \texttt{Carpet} does not run efficiently in
   parallel.



\section{The workhorse}

   While \texttt{Carpet} provides the necessary interfaces to the
   flesh, the grunt work is actually done by \texttt{CarpetLib}.  This
   thorn grew from an earlier mesh refinement of mine (Erik Schnetter)
   library that was independent of Cactus.  It has in the mean time
   been thoroughly changed, and it does not make sense any more to use
   it independent of Cactus.  \texttt{CarpetLib} contains of three
   major parts: a set of generic useful helpers, the grid hierarchy
   and data handling, and interpolation operators.  Especially the
   latter could probably be separated out.  While \texttt{CarpetLib}
   is written in C++, the interpolators are written in
   \textsc{Fortran77}.



\subsection{The helpers}

   The helpers correspond closely to Carpet's terminology.  A class
   \texttt{vect<T,D>} provides small \texttt{D}-dimensional vectors of
   the type \texttt{T}, with all the operators that one has learned to
   enjoy from Haskell and Fortran 90.  A \texttt{vect} corresponds to
   a grid point location.  The class \texttt{bbox<T,D>} provides
   \texttt{D}-dimensional bounding boxes using type \texttt{T} as
   indices.  A \texttt{bbox} defines the location and shape of a grid
   patch.  Finally, \texttt{bboxset<T,D>} is a collection of \texttt{bbox}es.
   \texttt{bboxsets} are a useful extension of the algebra of bboxes, as it
   closes the \texttt{bbox} algebra under the union operation.

   The files \texttt{CarpetLib/src/defs.*} defines useful small
   helpers and instantiates the STL templates.
   \texttt{CarpetLib/src/dist.*} provides some routines around MPI.
   Carpet is closely coupled to MPI and does not work without it.

   (Instead of inserting switches into Carpet to make it work without
   MPI, it would make more sense to use a dummy version of MPI.  PETSc
   does contain such a dummy version.  It is also easily possible to
   use a free MPI version such as MPICH and use that to run on a
   single processor.  However, I cannot see any real need for making
   Carpet work without MPI.)



\subsection{The grid hierarchy}

   The grid hierarchy is described by a set of classes.  Except for
   the actual data, all structures and all information is replicated
   on all processors.
\begin{description}
\item[\texttt{gh}]
   is a grid hierarchy.  It describes, for each refinement level, the
   location of the grid patches.  This \texttt{gh} does not contain
   ghost zones or prolongation boundaries.  There exists only one
   common \texttt{gh} for all grid functions.
\item[\texttt{dh}]
   is a data hierarchy.  It extends the notion of a \texttt{gh} by
   ghost zones and prolongation boundaries.  The \texttt{dh} does most
   of the bookkeeping work, deciding which grid patches interact with
   what other grid patches through synchronisation, prolongation,
   restriction, and boundary prolongation.  Unexpected situations are
   often caught in one of \texttt{dh}'s many self checks.  As all grid
   functions have the same number of ghost zones, there exists also
   only one \texttt{dh} for all grid functions.
\item[\texttt{th}]
   is a time hierarchy.  It extends the notion of a \texttt{gh} by
   multiple time levels.  There exists one \texttt{th} per grid
   function group.  This is a small class that keeps track of the
   current time on the different refinement levels.  (Note that
   different refinement levels usually live at different times.)
\item[\texttt{gf}]
   is a grid function of a certain variable type.  There is one
   instance of \texttt{gf} for every grid function, whether it has
   storage or not.  Each \texttt{gf} is associated with a \texttt{dh}
   and a \texttt{th} and holds the storage for all levels and all
   patches.  It provides interfaces to access and modify these data,
   either directly or through interpolation operators.  \texttt{gf}
   also handles the data movement during a regridding operation.
\item[\texttt{ggf}]
   is an abstract superclass of \texttt{gf} which is independent of
   the variable type.  This is necessary in C++ to prevent egregious
   code duplication due to class templates.  Most of the routines in
   \texttt{gf} are actually declared in \texttt{ggf}, and they either
   are virtual functions themselves, or they call virtual functions
   that are declared in \texttt{gf}.
\item[\texttt{data}]
   is a container for a grid patch of a certain variable type.  This is
   a glorified multi-dimensional array that knows how to move between
   processors.  \texttt{data} is not only used to store the grid
   patches that make up a \texttt{gf}, it is also used to move parts
   of patches around, e.g.\ for synchronisation or prolongation.
\item[\texttt{gdata}]
   is an abstract superclass of \texttt{data} for much the same
   reasons as for \texttt{ggf}.  All information that is independent
   of the variable type is kept in \texttt{gdata}.
\end{description}



\subsection{The interpolators}

   There are three kinds of ``interpolators'': for prolongation, for
   restricting, and for copying.  The latter is only a glorified
   hyperslabber that moves parts of grid patches between grid patches.

   The interpolators used for restriction and prolongation are
   different from those used for the generic interpolation interface
   in Cactus.  The reason is that interpolation is expensive, and
   hence the interpolation operators used for restriction and
   prolongation have to be streamlined and optimised.  As one knows
   the location of the sampling points for the interpolation, one can
   calculate the coefficients in advance, saving much time compared to
   calling a generic interpolation interface.



\subsubsection{Restriction}

   Restriction operators move data from finer to coarser grids.  They
   are typically called after both the coarse and the fine grid have
   been advanced to the same time, and they overwrite parts of the
   coarse grid with information from the fine grid, coupling the
   coarse grid evolution to the fine grid evolution.  In principle,
   there could be restriction operators with different orders of
   accuracy.  Currently only a single restriction operator is
   implemented that uses sampling.

   The interface of the restriction operator (see file
   \texttt{CarpetLib/src/restrict\_3d\_real8.F77}) is
\begin{verbatim}
subroutine restrict_3d_real8
      (src, srciext, srcjext, srckext,
       dst, dstiext, dstjext, dstkext,
       srcbbox, dstbbox, regbbox)

   integer    srciext, srcjext, srckext
   CCTK_REAL8 src(srciext,srcjext,srckext)
   integer    dstiext, dstjext, dstkext
   CCTK_REAL8 dst(dstiext,dstjext,dstkext)
   integer    srcbbox(3,3), dstbbox(3,3), regbbox(3,3)
\end{verbatim}
   This interpolator assumes that space has three dimensions.  The
   arrays \texttt{src} and \texttt{dst} contain the source (fine) and
   destination (coarse) grid patches, stored in Fortran order, as is
   customary in Cactus.  The arrays \texttt{src} and \texttt{dst} have
   the shapes \texttt{(srciext,srcjext,srckext)} and
   \texttt{(dstiext,dstjext,dstkext)}, respectively --- this
   corresponds to the \texttt{cctk\_lsh} field in the \texttt{cGH}
   structure.

   The three bboxes describe the location and shape of the two arrays
   and of the region that should be prolongated in the global grid
   point index system.  That is, while the two arrays \texttt{src} and
   \texttt{dst} are stored as dense arrays, they correspond to grid
   patches which in general have non-unit strides in the global index
   system.  As prolongation is an operation that is performed between
   overlapping grids, the prolongation region is the same for both the
   coarse and the fine grids.

   A few constraints must hold for these data.  For example, the
   shapes of the arrays must be the same as the shapes defined by the
   bounding boxes; the strides in the bounding boxes must differ by
   the refinement factor; the bounding boxes must overlap, and the
   region's bounding box must be contained in the arrays bounding
   boxes, etc.  Checking these constraints makes up about three
   quarters of the restriction routine.

   The bboxes themselves are here represented as Fortran arrays.
   Their meaning is
\begin{description}
\itemsep 0pt
\item[\texttt{bbox(:,1)}]
   lower boundary (inclusive)
\item[\texttt{bbox(:,2)}]
   upper boundary (inclusive)
\item[\texttt{bbox(:,3)}]
   stride
\end{description}



\subsubsection{Prolongation}

   There are many prolongation operators implemented.  They differ in
   the order of their interpolation in space (first and third, or
   linear and cubic interpolation) and in time (first and second, or
   linear and quadratic).  The higher the order of interpolation, the
   larger is the stencil, i.e.\ the more ghost zones and time levels
   are necessary, and the more expensive the operation becomes.

   The prolongation operators live in the files
   \texttt{CarpetLib/src/prolongate\_3d\_real8*.F77}, and the file
   names indicate their orders: \texttt{$n$tl} stands for $n$ time
   levels, and \texttt{o$n$} stands for an order $n$ interpolation in
   space (which uses a stencil that is $n+1$ grid points wide).

   Apart from taking more than one \texttt{src} array argument when
   using more than one time level, the interface to the prolongation
   operator is equivalent to that of the restriction operator
   described above.



\section{Regridding, how and where and when}

   The thorn \texttt{Carpet} provides a routine
   \texttt{RegisterRegridRoutine} where one can register a regridding
   routine.  Such a regridding routine does not have to actually
   regrid anything, it only has to return the new desired grid
   hierarchy, i.e.\ basically a description of a \texttt{gh}.

   Thorn \texttt{CarpetRegrid} provides a user interface to the
   regridding routines in Carpet.  All it does is take a regridding
   specification from the user and translate that into a \texttt{gh}.
   As usual, the parts where the computer has to listen to what a
   human being intends are the most complicated.

   As humans are usually more adept at getting used to computers than
   the other way around, it is useful and probably necessary to get
   acquainted with how Carpet thinks in order to make it do what is
   intended.

   Carpet does not deal with real-valued coordinates.  Carpet deals
   with integer grid point locations only, and it counts grid points
   in terms of the finest possible grid (not the finest currently
   existing grid).  The finest possible grid is defined by the maximum
   number of refinement levels set in \texttt{Carpet}.  Changing this
   parameter will change the meaning of many other values in parameter
   files, such as e.g.\ iteration numbers (termination, output).  The
   only parameter that is specified in terms of the coarsest grid is
   the shape of the coarsest grid in the \texttt{global\_*} parameters
   of \texttt{Carpet}.  I therefore suggest to set
   \texttt{max\_refinement\_levels} to some large number (e.g.\ 10),
   and then not changing it while experimenting with other parameter
   settings.

   Carpet also does not know about symmetries.  When specifying the
   location of a fine grid in terms of grid points, it is the
   responsibility of the user to place the fine grid correctly.  For
   that one has to take ghost zones and symmetry zones into account.

   It is also possible to specify the fine grid locations in terms of
   real-valued coordinates.  In this case, \texttt{CarpetRegrid}
   translates these into integer grid points.  A good translation is
   quite complicated, because it has to take many user expectations
   into account, such as the location of the origin, staggering with
   respect to the origin, symmetry boundary conditions, the number of
   ghost zones etc.  The current translation is naive and leads to
   unexpected results in many cases.  A routine that does most of the
   time what the user expects while being easy to understand is
   probably important for the ease of use of Carpet, but it might be
   some time until it is written.

   \texttt{CarpetRegrid} contains also a routine that measures the
   error, as provided in a grid function, and the automatically
   decides where to refine.  This is called AMR (adaptive mesh
   refinement) if it works efficiently.

   Much of \texttt{CarpetRegrid} is just slabbed together in an
   attempt to find out what people need and expect.  The thorn is a
   mess, and a complete rewrite might be a good idea, once one knows
   what exactly the rewritten thorn should do.



\section{Random ramblings}

   Carpet uses the STL, because the STL provides very useful container
   classes such as vectors, sets, and lists.  Writing these abstract
   datatypes oneself does not make sense in these times.  It makes
   much more sense to politick computer administrators to upgrade
   their software.

   The STL and \texttt{CarpetLib}'s classes need to be instantiated
   explicitly.  Several compilers have several ``automatic'' schemes
   that handle all template issues ``just fine''.  Except they don't.
   One wants to select the following: No automatic inclusion of
   \texttt{.cc} files, no automatic template instantiation at link
   time.  Instead, most templates are instantiated explicitly by
   \texttt{CarpetLib}.  It is also necessary to specify to instantiate
   used templates automatically.  The explicit instantiations of
   \texttt{CarpetLib}'s classes live in the \texttt{.cc} files
   corresponding to the \texttt{.hh} file that define the templates.
   The STL templates are instantiated in the file
   \texttt{CarpetLib/src/defs.cc}.

   Carpet makes extensive use of the \texttt{assert()} macro in C.
   This is a quick and easy way to ensure that a certain condition
   holds.  Assert statements abort the code if the condition does not
   hold.  Although I try to provide useful error messages to the user,
   many unexpected cases are only caught deep inside Carpet and
   manifest themselves as assertion failures.  If you report an
   assertion failure, it is vitally important to remember
   theaccompanying file name and line number.  It would also be useful
   to extract from the core file a stack backtrace and the values of
   the local variables of the current stack frame.

   Using symmetry boundary conditions such as octant mode is currently
   still awkward in Carpet.  There are several reasons for this:
   \texttt{CarpetRegrid} does not know about symmetries, and hence
   doesn't take them into account when choosing refinement regions.
   The symmetry conditions on the finer grid might be different from
   the conditions on the coarser grids, and the symmetry thorns cannot
   cope with this, so this situation must be avoided: one cannot use
   \texttt{avoid\_origin=yes}, because the finer grids all have
   \texttt{avoid\_origin=no} due to the vertex-centred refinement.

\end{document}
